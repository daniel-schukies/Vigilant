Das Programm ist in vier Lua-Skripten implementiert.\\
Durch die Konfigurationsvariablen, die in \textit{config.lua} hinterlegt sind, werden die verschiedenen Parameter des Programms festgelegt:
\begin{center}
\begin{tabular}{|l|l|}
     \hline
     \textbf{Variable} & \textbf{Bedeutung} \\ \hline \hline
     KeepVideoDays & Videodateien werden nach dieser Anzahl Tage gel\"oscht.  \\ \hline
     MySQLDBHostName & Hostname des Datenbank-Servers \\ \hline
     MySQLDBHostPort & Host-Port des Datenbank-Servers \\ \hline
     MySQLDBUserName & Zu verwendender Datenbank-Nutzer \\ \hline
     MySQLDBUserPassword & Passwort f\"ur den Datenbank-Nutzer \\ \hline
     RecordStopDelay & Die Aufnahme l\"auft f\"ur diese Anzahl Sekunden nach. \\ \hline
     SensorInputPin & An diesem Pin liegt das Schaltsignal des Sensors an. \\ \hline
     VideoCaptureFramerate & Das Video wird mit dieser Bildwiederholrate aufgenommen. \\ \hline
     VideoCaptureHeight & Pixelh\"ohe des Videos \\ \hline
     VideoCaptureWidth & Pixebreite des Videos \\ \hline
     VideoDeviceMount & Linux Device File der Kamera \\ \hline
     VideoDeviceStream & Der aufzunehmende Stream der Kamera. \\ \hline
     VideoOutputDir & Das Video wird in diesem Pfad gespeichert. \\ \hline
\end{tabular}
\end{center}
Diese Variablen sind für alle anderen Lua-Skripte sichtbar.\\
Die Kamera muss die gew\"ahlten Parameter unterst\"utzen, ansonsten f\"allt die Aufnahme auf die Standardwerte des VLC Players zur\"uck.\\

Das Skript \textit{init.lua} wird beim Start des Programms einmalig ausgef\"uhrt und initialisiert zuerst das luaposix-Modul zur weiteren Benutzung. Anschlie{\ss}end wird das LuaSQL-Modul initialisert und eine Verbindung zur Datenbank aufgebaut. Au{\ss}erdem wird die GPIO-Schnittstelle initialisiert und so konfiguriert, dass vom festgelegten Pin das Signal des Sensors gelesen werden kann.\\

Die Kernfunktionalit\"at des Programms ist in \textit{record.lua} definiert.
Das Programm l\"auft in einer Endlosschleife; zum Beenden gen\"ugt es,  ein SIGINT-Signal an den Prozess zu senden\footnote{Selber Effekt wie  die  Tastenkombination Ctrl-C im Terminal}. Der Ablauf eines Schleifen-Durchlaufs wird im folgenden beschrieben.\\
\\
Zun\"achst wird der Zustand des Sensors (1-ge\"offnet, 0-geschlossen) abgefragt und gespeichert, der vorherige Zustand wird in einer anderen Variable behalten. Au{\ss}erdem werden Systemzeit und Datum tempor\"ar gespeichert - diese werden im sp\"ateren Verlauf ben\"otigt.
Als n\"achstes werden abh\"angig vom Sensor-Zustand verschiedene Aktionen durchgef\"uhrt.\\
\\
Ist der Zustand des Sensors '1' und der vorherige Zustand war nicht '1' (das bedeutet, die T\"ur wurde ge\"offnet), ist der Ablauf wie folgt:
Wenn die Aufnahme nicht noch vom letzten T\"ur-\"Offnen l\"auft, wird eine neue Aufnahme gestartet. Dazu wird mit der POSIX-Standardfunktion fork() ein neuer Prozess erstellt und eine Instanz des VLC-Players geladen, welche eine neue Video-Datei im konfigurierten Verzeichnis erstellt. Der Dateiname der Aufnahme setzt sich dabei einfach zusammen aus Datum und Uhrzeit des Aufnahmestarts (hat also das Format 'yyyy-mm-dd-HHMMSS'.mp4).
Falls die Aufnahme noch nachl\"auft, wenn die T\"ur ge\"offnet wird, wird die laufende Aufnahme fortgesetzt, der Aufnahmestopp-Timer wird angehalten.
In jedem Fall wird daraufhin ein neuer Datenbank-Eintrag mit Zeitstempel und Video-Dateiname erstellt.\\
\\
Ist der Zustand des Sensors '0' und der vorherige Zustand war nicht '0' (die T\"ur wurde geschlossen), geschieht folgendes:
Vorausgesetzt, dass die Aufnahme l\"auft (was normalerweise der Fall sein sollte), wird oben ew\"ahnter Timer gestartet, der f\"ur das Nachlaufen der Aufnahme um die konfigurierte Anzahl Sekunden zust\"andig ist.\\
\\
Am Ende der Schleife wird noch der Timer aktualisiert, falls er l\"auft.
Wenn der Timer die eingestellte Nachlaufzeit \"uberschritten hat, wird die Aufnahme durch das senden eines Interrupt-Signals an den VLC Player gestoppt.\\
\\

Eine Routine zum L\"oschen alter Video-Dateien ist in \textit{dbclean.lua} implementiert. Dabei wird die Datenbank nach Eintr\"agen durchsucht, die \"alter sind als die eingestellte Anzahl Tage. Anschlie{\ss}end werden die referenzierten Video-Dateien gel\"oscht und die Eintr\"age mit dem Wert NULL als Video-Dateiname aktualisiert. Dadurch erkennt das Web-Interface, dass das Video gel\"oscht wurde.
\vspace{1.5cm}