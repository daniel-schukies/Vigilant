Die Datenbank muss folgende Informationen \"uber eine Aufnahme speichern:
\begin{itemize}
    \item Datum der Aufnahme
    \item Genaue Uhrzeit zu Beginn der Augnahme
    \item Dateiname des Videos (um es sp\"ater anzeigen zu k\"onnen)
\end{itemize}
\vspace{0.5cm}
Daraus ergibt sich ein recht simples Tabellenformat:\\

\begin{lstlisting}[language=SQL,caption={Auszug aus dbRecords.sql},numbers=left,frame=lrbt]
CREATE TABLE Records (
    EntryID INTEGER UNIQUE NOT NULL AUTO_INCREMENT,
    RecordDateTime TIMESTAMP DEFAULT CURRENT_TIMESTAMP,
    VideoPath VARCHAR(4096) DEFAULT NULL,
    PRIMARY KEY (EntryID)
)  CHARSET=UTF8;
\end{lstlisting}
\vspace{0.5cm}
Datum und Uhrzeit der Aufnahme werden in RecordDateTime gespeichert.
Der Datentyp TIMESTAMP hat das Format 'yyyy-mm-dd HH:MM:SS'.
Der Dateiname des Videos wird in der Variable VideoPath gespeichert (wider der Namengebung wird tats\"achlich nur der Name der Datei gespeichert; der Pfad zu den Videodateien ist dem Webdienst bekannt).\\
\\
Weiterhin wird ein SQL-Nutzer mit ausreichenden Privilegien ben\"otigt.\\
\begin{lstlisting}[language=SQL,caption={Anlegen des Datenbank-Nutzers},numbers=left,frame=lrbt]
CREATE USER 'phpwebuser'@'localhost'
    IDENTIFIED BY 'secure_password';
\end{lstlisting}
\vspace{0.5cm}
Der Nutzer muss in diesem Szenario nur lokal erreichbar sein.
Skript und Webdienst m\"ussen sp\"ater Eintr\"age lesen, schreiben, ver\"andern und l\"oschen k\"onnen. Es werden also entsprechend Rechte vergeben:\\
\newpage
\begin{lstlisting}[language=SQL,caption={Die Rechte des Datenbank-Nutzers},numbers=left,frame=lrbt]
GRANT USAGE,SELECT,INSERT,UPDATE,DELETE 
    ON Records.* 
    TO 'phpwebuser'@localhost';
\end{lstlisting}
\vspace{0.5cm}
Die Datenbank ist nun einsatzbereit.
\vspace{1.5cm}