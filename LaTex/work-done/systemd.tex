Damit \textit{dbclean.lua} automatisch in regelm\"a{\ss}igen Abst\"anden aufgerufen wird, existieren ein systemd-Skript \textit{dbclean.service} und ein systemd-Timer \textit{dbclean.timer}. Die systemd-Units haben den Vorteil, dass durch sie die Lua-Skripte  einfach und schnell aktiviert und deaktiviert werden k\"onnen. Wenn der Timer aktiviert ist, ruft er \textit{dbclean.lua} in regelm\"a{\ss}igen Abst\"anden auf (konfiguriert ist t\"aglich um 00:00 Uhr).\\
\\
Au{\ss}erdem existiert noch eine Unit \textit{record.service}, welche verwendet wird, um \textit{record.lua} im Hintergrund auszuf\"uhren und zu steuern.\\

\begin{lstlisting}[caption={Ein systemd-Timer hat eine relativ einfache Struktur.},numbers=left,frame=lrbt]
[Unit]
Description=Lua clean script timer
Requires=mysql.service

[Timer]
OnCalendar=daily
Unit=dbclean.service

[Install]
WantedBy=multi-user.target
\end{lstlisting}
\vspace{1.5cm}