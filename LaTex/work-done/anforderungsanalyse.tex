\chapter{Anforderungsanalyse}

\section{Priorisierung}

\subsection{Muss}
\begin{itemize}
    \item Der Zustand der T\"ur (Offen/Zu) wird von einem Magnetschalter erfasst, welcher \"uber die GPIO-Schnitstelle mit dem Raspberry Pi verbunden ist.
    \item Die Kamera ist \"uber USB mit dem Raspberry Pi verbunden.
    \item Das System kann per WLAN \"uber einen USB-Adapter mit dem Netzwerk verbunden werden, um Installationen ohne Ehternet-Schnittstelle zu erm\"oglichen.
    \item Die Kamera muss so positioniert werden, dass das Gesicht der Person, die durch die T\"ur tritt, erkennbar ist.
    \item Die Kabel m\"ussen lang genug sein, um das System auch an gr\"o{\ss}eren Zug\"angen installieren zu k\"onnen.
    \item Auf der Hardware muss ein LAMP-Server sowie ein Datenbank-Server lauff\"ahig sein.
\end{itemize}

\subsection{Soll}
\begin{itemize}
    \item Magnet und Magnetschalter m\"ussen so an der T\"ur anbringbar sein, dass man den Mechanismus nicht umgehen kann.
    \item Das System soll stabil genug laufen, um im Dauermodus betrieben werden zu k\"onnen.
    \item Der Datenbank-Eintrag soll zu Beginn der Aufnahme geschrieben werden, damit wenigstens die Zeiterfassung funktioniert (falls Kamera ausf\"allt).
    \item Ein systemd-Service startet das Script zum Systemstart und initialisiert die Hardware.
    \item Ein systemd-Timer durchsucht im definierten Zeit-Intervall die Datenbank nach Eintr\"agen, die die definierte Speicherungszeit \"uberschritten haben, und entfernt die Video-Dateien. Die Datenbank-Eintr\"age bleiben bestehen; diese werden \"uber das Web-Interface gel\"oscht.
    \item Button zum L\"oschen aller alten Logeintr\"age in der Datenbank.
\end{itemize}

\subsection{Kann}
\label{sec:kann}
\begin{itemize}
    \item Aktivierung des Systems durch T\"urklingel
    \item Platzierung der Kamera auf Motorplatte zur Erweiterung des \"Uberwachungsbreichs
    \item Unterst\"utzung f\"ur dedizierte Datenbank-Server/Videospeicher
    \item Ton-Aufzeichnung
    \item Gesichtserkennung von eingehenden Personen
    \item Live-Stream der Kamera \"uber Web-Interface
\end{itemize}

\newpage