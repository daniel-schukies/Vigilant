\subsubsection{LAMP}
F\"ur das Web-Interface wird ein LAMP\footnote{LAMP(Linux-Apache-MySQL-PHP) ist ein gängiges Server-Setup}-Server ben\"otigt.\\
Installiert werden:
\begin{itemize}
    \item Apache HTTP Server Version 2.4
    \item PHP 5.6.28
    \item MariaDB Server 10.0
\end{itemize}
\vspace{0.5cm}
\subsubsection{VLC Media Player}
Der VLC Media Player\footnote{Projektseite: \url{https://www.videolan.org/vlc/}} bietet eine komfortable M\"oglichkeit, Video-Ger\"ate \"uber den Video4Linux-Treiber (V4L) anzusprechen und deren Video-Stream in einem Video-Container zu speichern. Installiert wird nur das Kommandozeilen-Frontend - die grafische Oberfl\"ache wird nicht ben\"otigt.
\\
\subsubsection{Lua}
Installiert wird aus Kompatibilit\"atsgr\"unden Lua 5.1\footnote{Aktuelle Version ist 5.3, Debian-Repository-Version ist 5.2}.\\
\\
Au{\ss}erdem sind noch einige Module, die Zusatz-Funktionen bereitstellen, erforderlich.\\
\\
Folgende Module werden installiert:

\begin{itemize}
    \item LuaSQL 2.3 (Datenbank-Adapter)
    \item luaposix 33.4.0 (erm\"oglicht Nutzung einiger POSIX-Funktionen)
    \item bit32 5.3 (Bitmanipulationen, ben\"otigt von luaposix)
\end{itemize}

\newpage